% Generated by Sphinx.
\def\sphinxdocclass{report}
\newif\ifsphinxKeepOldNames \sphinxKeepOldNamestrue
\documentclass[letterpaper,10pt,english]{sphinxmanual}
\usepackage{iftex}

\ifPDFTeX
  \usepackage[utf8]{inputenc}
\fi
\ifdefined\DeclareUnicodeCharacter
  \DeclareUnicodeCharacter{00A0}{\nobreakspace}
\fi
\usepackage{cmap}
\usepackage[T1]{fontenc}
\usepackage{amsmath,amssymb,amstext}
\usepackage{babel}
\usepackage{times}
\usepackage[Bjarne]{fncychap}
\usepackage{longtable}
\usepackage{sphinx}
\usepackage{multirow}
\usepackage{eqparbox}


\addto\captionsenglish{\renewcommand{\figurename}{Fig.\@ }}
\addto\captionsenglish{\renewcommand{\tablename}{Table }}
\SetupFloatingEnvironment{literal-block}{name=Listing }

\addto\extrasenglish{\def\pageautorefname{page}}

\setcounter{tocdepth}{1}


\title{Sec9 Documentation}
\date{Oct 26, 2016}
\release{1.0}
\author{Yuting Ye}
\newcommand{\sphinxlogo}{}
\renewcommand{\releasename}{Release}
\makeindex

\makeatletter
\def\PYG@reset{\let\PYG@it=\relax \let\PYG@bf=\relax%
    \let\PYG@ul=\relax \let\PYG@tc=\relax%
    \let\PYG@bc=\relax \let\PYG@ff=\relax}
\def\PYG@tok#1{\csname PYG@tok@#1\endcsname}
\def\PYG@toks#1+{\ifx\relax#1\empty\else%
    \PYG@tok{#1}\expandafter\PYG@toks\fi}
\def\PYG@do#1{\PYG@bc{\PYG@tc{\PYG@ul{%
    \PYG@it{\PYG@bf{\PYG@ff{#1}}}}}}}
\def\PYG#1#2{\PYG@reset\PYG@toks#1+\relax+\PYG@do{#2}}

\expandafter\def\csname PYG@tok@gd\endcsname{\def\PYG@tc##1{\textcolor[rgb]{0.63,0.00,0.00}{##1}}}
\expandafter\def\csname PYG@tok@gu\endcsname{\let\PYG@bf=\textbf\def\PYG@tc##1{\textcolor[rgb]{0.50,0.00,0.50}{##1}}}
\expandafter\def\csname PYG@tok@gt\endcsname{\def\PYG@tc##1{\textcolor[rgb]{0.00,0.27,0.87}{##1}}}
\expandafter\def\csname PYG@tok@gs\endcsname{\let\PYG@bf=\textbf}
\expandafter\def\csname PYG@tok@gr\endcsname{\def\PYG@tc##1{\textcolor[rgb]{1.00,0.00,0.00}{##1}}}
\expandafter\def\csname PYG@tok@cm\endcsname{\let\PYG@it=\textit\def\PYG@tc##1{\textcolor[rgb]{0.25,0.50,0.56}{##1}}}
\expandafter\def\csname PYG@tok@vg\endcsname{\def\PYG@tc##1{\textcolor[rgb]{0.73,0.38,0.84}{##1}}}
\expandafter\def\csname PYG@tok@vi\endcsname{\def\PYG@tc##1{\textcolor[rgb]{0.73,0.38,0.84}{##1}}}
\expandafter\def\csname PYG@tok@mh\endcsname{\def\PYG@tc##1{\textcolor[rgb]{0.13,0.50,0.31}{##1}}}
\expandafter\def\csname PYG@tok@cs\endcsname{\def\PYG@tc##1{\textcolor[rgb]{0.25,0.50,0.56}{##1}}\def\PYG@bc##1{\setlength{\fboxsep}{0pt}\colorbox[rgb]{1.00,0.94,0.94}{\strut ##1}}}
\expandafter\def\csname PYG@tok@ge\endcsname{\let\PYG@it=\textit}
\expandafter\def\csname PYG@tok@vc\endcsname{\def\PYG@tc##1{\textcolor[rgb]{0.73,0.38,0.84}{##1}}}
\expandafter\def\csname PYG@tok@il\endcsname{\def\PYG@tc##1{\textcolor[rgb]{0.13,0.50,0.31}{##1}}}
\expandafter\def\csname PYG@tok@go\endcsname{\def\PYG@tc##1{\textcolor[rgb]{0.20,0.20,0.20}{##1}}}
\expandafter\def\csname PYG@tok@cp\endcsname{\def\PYG@tc##1{\textcolor[rgb]{0.00,0.44,0.13}{##1}}}
\expandafter\def\csname PYG@tok@gi\endcsname{\def\PYG@tc##1{\textcolor[rgb]{0.00,0.63,0.00}{##1}}}
\expandafter\def\csname PYG@tok@gh\endcsname{\let\PYG@bf=\textbf\def\PYG@tc##1{\textcolor[rgb]{0.00,0.00,0.50}{##1}}}
\expandafter\def\csname PYG@tok@ni\endcsname{\let\PYG@bf=\textbf\def\PYG@tc##1{\textcolor[rgb]{0.84,0.33,0.22}{##1}}}
\expandafter\def\csname PYG@tok@nl\endcsname{\let\PYG@bf=\textbf\def\PYG@tc##1{\textcolor[rgb]{0.00,0.13,0.44}{##1}}}
\expandafter\def\csname PYG@tok@nn\endcsname{\let\PYG@bf=\textbf\def\PYG@tc##1{\textcolor[rgb]{0.05,0.52,0.71}{##1}}}
\expandafter\def\csname PYG@tok@no\endcsname{\def\PYG@tc##1{\textcolor[rgb]{0.38,0.68,0.84}{##1}}}
\expandafter\def\csname PYG@tok@na\endcsname{\def\PYG@tc##1{\textcolor[rgb]{0.25,0.44,0.63}{##1}}}
\expandafter\def\csname PYG@tok@nb\endcsname{\def\PYG@tc##1{\textcolor[rgb]{0.00,0.44,0.13}{##1}}}
\expandafter\def\csname PYG@tok@nc\endcsname{\let\PYG@bf=\textbf\def\PYG@tc##1{\textcolor[rgb]{0.05,0.52,0.71}{##1}}}
\expandafter\def\csname PYG@tok@nd\endcsname{\let\PYG@bf=\textbf\def\PYG@tc##1{\textcolor[rgb]{0.33,0.33,0.33}{##1}}}
\expandafter\def\csname PYG@tok@ne\endcsname{\def\PYG@tc##1{\textcolor[rgb]{0.00,0.44,0.13}{##1}}}
\expandafter\def\csname PYG@tok@nf\endcsname{\def\PYG@tc##1{\textcolor[rgb]{0.02,0.16,0.49}{##1}}}
\expandafter\def\csname PYG@tok@si\endcsname{\let\PYG@it=\textit\def\PYG@tc##1{\textcolor[rgb]{0.44,0.63,0.82}{##1}}}
\expandafter\def\csname PYG@tok@s2\endcsname{\def\PYG@tc##1{\textcolor[rgb]{0.25,0.44,0.63}{##1}}}
\expandafter\def\csname PYG@tok@nt\endcsname{\let\PYG@bf=\textbf\def\PYG@tc##1{\textcolor[rgb]{0.02,0.16,0.45}{##1}}}
\expandafter\def\csname PYG@tok@nv\endcsname{\def\PYG@tc##1{\textcolor[rgb]{0.73,0.38,0.84}{##1}}}
\expandafter\def\csname PYG@tok@s1\endcsname{\def\PYG@tc##1{\textcolor[rgb]{0.25,0.44,0.63}{##1}}}
\expandafter\def\csname PYG@tok@ch\endcsname{\let\PYG@it=\textit\def\PYG@tc##1{\textcolor[rgb]{0.25,0.50,0.56}{##1}}}
\expandafter\def\csname PYG@tok@m\endcsname{\def\PYG@tc##1{\textcolor[rgb]{0.13,0.50,0.31}{##1}}}
\expandafter\def\csname PYG@tok@gp\endcsname{\let\PYG@bf=\textbf\def\PYG@tc##1{\textcolor[rgb]{0.78,0.36,0.04}{##1}}}
\expandafter\def\csname PYG@tok@sh\endcsname{\def\PYG@tc##1{\textcolor[rgb]{0.25,0.44,0.63}{##1}}}
\expandafter\def\csname PYG@tok@ow\endcsname{\let\PYG@bf=\textbf\def\PYG@tc##1{\textcolor[rgb]{0.00,0.44,0.13}{##1}}}
\expandafter\def\csname PYG@tok@sx\endcsname{\def\PYG@tc##1{\textcolor[rgb]{0.78,0.36,0.04}{##1}}}
\expandafter\def\csname PYG@tok@bp\endcsname{\def\PYG@tc##1{\textcolor[rgb]{0.00,0.44,0.13}{##1}}}
\expandafter\def\csname PYG@tok@c1\endcsname{\let\PYG@it=\textit\def\PYG@tc##1{\textcolor[rgb]{0.25,0.50,0.56}{##1}}}
\expandafter\def\csname PYG@tok@o\endcsname{\def\PYG@tc##1{\textcolor[rgb]{0.40,0.40,0.40}{##1}}}
\expandafter\def\csname PYG@tok@kc\endcsname{\let\PYG@bf=\textbf\def\PYG@tc##1{\textcolor[rgb]{0.00,0.44,0.13}{##1}}}
\expandafter\def\csname PYG@tok@c\endcsname{\let\PYG@it=\textit\def\PYG@tc##1{\textcolor[rgb]{0.25,0.50,0.56}{##1}}}
\expandafter\def\csname PYG@tok@mf\endcsname{\def\PYG@tc##1{\textcolor[rgb]{0.13,0.50,0.31}{##1}}}
\expandafter\def\csname PYG@tok@err\endcsname{\def\PYG@bc##1{\setlength{\fboxsep}{0pt}\fcolorbox[rgb]{1.00,0.00,0.00}{1,1,1}{\strut ##1}}}
\expandafter\def\csname PYG@tok@mb\endcsname{\def\PYG@tc##1{\textcolor[rgb]{0.13,0.50,0.31}{##1}}}
\expandafter\def\csname PYG@tok@ss\endcsname{\def\PYG@tc##1{\textcolor[rgb]{0.32,0.47,0.09}{##1}}}
\expandafter\def\csname PYG@tok@sr\endcsname{\def\PYG@tc##1{\textcolor[rgb]{0.14,0.33,0.53}{##1}}}
\expandafter\def\csname PYG@tok@mo\endcsname{\def\PYG@tc##1{\textcolor[rgb]{0.13,0.50,0.31}{##1}}}
\expandafter\def\csname PYG@tok@kd\endcsname{\let\PYG@bf=\textbf\def\PYG@tc##1{\textcolor[rgb]{0.00,0.44,0.13}{##1}}}
\expandafter\def\csname PYG@tok@mi\endcsname{\def\PYG@tc##1{\textcolor[rgb]{0.13,0.50,0.31}{##1}}}
\expandafter\def\csname PYG@tok@kn\endcsname{\let\PYG@bf=\textbf\def\PYG@tc##1{\textcolor[rgb]{0.00,0.44,0.13}{##1}}}
\expandafter\def\csname PYG@tok@cpf\endcsname{\let\PYG@it=\textit\def\PYG@tc##1{\textcolor[rgb]{0.25,0.50,0.56}{##1}}}
\expandafter\def\csname PYG@tok@kr\endcsname{\let\PYG@bf=\textbf\def\PYG@tc##1{\textcolor[rgb]{0.00,0.44,0.13}{##1}}}
\expandafter\def\csname PYG@tok@s\endcsname{\def\PYG@tc##1{\textcolor[rgb]{0.25,0.44,0.63}{##1}}}
\expandafter\def\csname PYG@tok@kp\endcsname{\def\PYG@tc##1{\textcolor[rgb]{0.00,0.44,0.13}{##1}}}
\expandafter\def\csname PYG@tok@w\endcsname{\def\PYG@tc##1{\textcolor[rgb]{0.73,0.73,0.73}{##1}}}
\expandafter\def\csname PYG@tok@kt\endcsname{\def\PYG@tc##1{\textcolor[rgb]{0.56,0.13,0.00}{##1}}}
\expandafter\def\csname PYG@tok@sc\endcsname{\def\PYG@tc##1{\textcolor[rgb]{0.25,0.44,0.63}{##1}}}
\expandafter\def\csname PYG@tok@sb\endcsname{\def\PYG@tc##1{\textcolor[rgb]{0.25,0.44,0.63}{##1}}}
\expandafter\def\csname PYG@tok@k\endcsname{\let\PYG@bf=\textbf\def\PYG@tc##1{\textcolor[rgb]{0.00,0.44,0.13}{##1}}}
\expandafter\def\csname PYG@tok@se\endcsname{\let\PYG@bf=\textbf\def\PYG@tc##1{\textcolor[rgb]{0.25,0.44,0.63}{##1}}}
\expandafter\def\csname PYG@tok@sd\endcsname{\let\PYG@it=\textit\def\PYG@tc##1{\textcolor[rgb]{0.25,0.44,0.63}{##1}}}

\def\PYGZbs{\char`\\}
\def\PYGZus{\char`\_}
\def\PYGZob{\char`\{}
\def\PYGZcb{\char`\}}
\def\PYGZca{\char`\^}
\def\PYGZam{\char`\&}
\def\PYGZlt{\char`\<}
\def\PYGZgt{\char`\>}
\def\PYGZsh{\char`\#}
\def\PYGZpc{\char`\%}
\def\PYGZdl{\char`\$}
\def\PYGZhy{\char`\-}
\def\PYGZsq{\char`\'}
\def\PYGZdq{\char`\"}
\def\PYGZti{\char`\~}
% for compatibility with earlier versions
\def\PYGZat{@}
\def\PYGZlb{[}
\def\PYGZrb{]}
\makeatother

\renewcommand\PYGZsq{\textquotesingle}

\begin{document}

\maketitle
\tableofcontents
\phantomsection\label{index::doc}


Contents:


\chapter{Problem 1 (Section 9.11 Problem 23)}
\label{P1:stat135-section-9}\label{P1::doc}\label{P1:problem-1-section-9-11-problem-23}\label{P1:problem1}
Suppose that a \(99\%\) confidence interval for the mean \(\mu\) of a normal distribution is found to be \((-2.0, 3.0)\). Would a test of \(H_0: \mu = -3\) versus \(H_A: \mu \neq -3\) be rejected at the .01 significance level?

Solution:

Suppose the confidence interval is \(C(X)\), and the corresponding acceptance
rejion is \(A(\mu)\). By the duality of confidence interval and hypothsis testing, we have
\begin{quote}

\(\mu \in C(X) \Leftrightarrow X \in A(\mu)\)
\end{quote}

Thus, the \(99\%\) confidence interval corresponds to that the hypothesis testthat accepts the null hypothesis with a probability of \(99\%\) given \(H_0\), or the type I error is \(1 - 99\% = 1\%\), which means the significance level of the test is \(0.01\).

However, \(-3 \not \in (-2.0, 3.0)\), which implies that the null is rejected.Hence, the test of \(H_0: \mu = -3\) versus \(H_A: \mu \neq -3\) is rejected at the \(0.01\) significance level.


\chapter{Problem 2 (Section 9.11 Problem 39)}
\label{P2:problem-2-section-9-11-problem-39}\label{P2::doc}\label{P2:problem2}
Thereisagreatdealoffolkloreabouttheeffectsofthefullmoononhumansand other animals. Do animals bite humans more during a full moon? In an attempt to study this question, Bhattacharjee et al. (2000) collected data on admissions to a medical facility for treatment of bites by animals: cats, rats, horses, and dogs. \(95\%\) of the bites were by man’s best friend, the dog. The lunar cycle was divided into 10 periods, and the number of bites in each period is shown in the following table. Day 29 is the full moon. Is there a temporal trend in the incidence of bites?

\noindent\begin{tabulary}{\linewidth}{|L|L|L|L|L|L|L|L|L|L|L|}
\hline
\textsf{\relax 
Lunar Day
\unskip}\relax &\textsf{\relax 
16-18
\unskip}\relax &\textsf{\relax 
19-21
\unskip}\relax &\textsf{\relax 
22-24
\unskip}\relax &\textsf{\relax 
25-27
\unskip}\relax &\textsf{\relax 
28-1
\unskip}\relax &\textsf{\relax 
2- 4
\unskip}\relax &\textsf{\relax 
5- 7
\unskip}\relax &\textsf{\relax 
8-10
\unskip}\relax &\textsf{\relax 
11-13
\unskip}\relax &\textsf{\relax 
14-15
\unskip}\relax \\
\hline
Number of Bites
&
137
&
150
&
163
&
201
&
269
&
155
&
142
&
146
&
148
&
110
\\
\hline\end{tabulary}


Solution:

\noindent\sphinxincludegraphics{{Lunar_Bites}.png}

From the figure above, we can conclude that animal bite human more during the
full moon days.


\chapter{Problem 3 (Section 9.11 Problem 41)}
\label{P3:problem3}\label{P3:problem-3-section-9-11-problem-41}\label{P3::doc}
Let \(X_i \sim bin(n_i, p_i)\), for \(i=1,\dots,m\), be independent. Derive a likelihood ratio test for the hypothesis

\(H_0: p_1 = p_2 = \dots = p_m\)

against the alternative hypothesis that the \(p_i\) are not all equal. What is the large sample distribution of the test statistic?

Solution:

The generalized likelihood ratio is

\(\Lambda(x) = \frac{\max_{H_0} \prod_{i=1}^m {n_i \choose x_i} p_i^{x_i}(1 - p_i)^{n_i - x_i}}{\max \prod_{i=1}^m {n_i \choose x_i} p_i^{x_i}(1 - p_i)^{n_i - x_i}}\)

It's easy to show that under the null hypothesis, the MLE is \(\hat{p} = \frac{\sum_{i=1}^m x_i}{\sum_{i=1}^n n_i}\) while the alternative hypothsis, the MLEs are \(\hat{p}_i = \frac{x_i}{n_i}\), for \(i = 1, \dots, m\). Then

\(\max_{H_0} \prod_{i=1}^m {n_i \choose x_i} p_i^{x_i}(1 - p_i)^{n_i - x_i} = \prod_{i=1}^m {n_i \choose x_i} \hat{p}^{x_i} (1 - \hat{p})^{n_i - x_i} =  \hat{p}^{\sum_{i=1}^n n_i \hat{p}_i} (1 - \hat{p})^{\sum_{i=1}^m n_i(1 - \hat{p}_i)} \left(\prod_{i=1}^m {n_i \choose x_i}\right)\)

and

\(\max \prod_{i=1}^m {n_i \choose x_i} p_i^{x_i}(1 - p_i)^{n_i - x_i} = \prod_{i=1}^m {n_i \choose x_i} \hat{p}_i^{x_i} (1 - \hat{p}_i)^{n_i - x_i} = \left(\prod_{i=1}^m {n_i \choose x_i}\right) \times \left( \prod_{i=1}^m \hat{p}_i^{n_i \hat{p}_i} (1 - \hat{p}_i)^{n_i (1 - \hat{p}_i)}\right)\)

Thus

\(\Lambda(x) = \frac{\hat{p}^{\sum_{i=1}^m n_i \hat{p}_i} \times (1 - \hat{p})^{\sum_{i=1}^m n_i(1 - \hat{p}_i)}}{\prod_{i=1}^m (\hat{p}_i)^{n_i \hat{p}_i} (1 - \hat{p}_i)^{n_i(1 - \hat{p}_i)}}\)

Then we have

\(-2\log (\Lambda) = 2\log \left\{\frac{\prod_{i=1}^m (\hat{p}_i)^{n_i \hat{p}_i} (1 - \hat{p}_i)^{n_i(1 - \hat{p}_i)}}{\hat{p}^{\sum_{i=1}^m n_i \hat{p}_i} \times (1 - \hat{p})^{\sum_{i=1}^m n_i(1 - \hat{p}_i)}}\right\}\)

By the same argument as Section 9.5 (Taylor Expansion), we have

\(-2\log(\Lambda) \approx \sum_{i=1}^m \frac{(x_i - n_i \hat{p})^2}{n_i \hat{p} (1 - \hat{p})}\)

which follows a chi-squre distribution with \(df = m - 1\) degrees of freedom.


\chapter{Problem 4 (Section 9.11 Problem 42)}
\label{P4:problem4}\label{P4::doc}\label{P4:problem-4-section-9-11-problem-42}
Nylon bars were tested for brittleness (Bennett and Franklin 1954). Each of 280 bars was molded under similar conditions and was tested in five places. Assuming that each bar has uniform composition, the number of breaks on a given bar should be binomially distributed with five trials and an unknown probability p of failure. If the bars are all of the same uniform strength, p should be the same for all of them; if they are of different strengths, p should vary from bar to bar. Thus, the null hypothesis is that the p’s are all equal. The following table summarizes the outcome of the experiment:

\noindent\begin{tabulary}{\linewidth}{|L|L|}
\hline
\textsf{\relax 
Breaks/Bar
\unskip}\relax &\textsf{\relax 
Frequency
\unskip}\relax \\
\hline
0
&
157
\\
\hline
1
&
69
\\
\hline
2
&
35
\\
\hline
3
&
17
\\
\hline
4
&
1
\\
\hline
5
&
1
\\
\hline\end{tabulary}

\begin{enumerate}
\item {} 
Under the given assumption, the data in the table consist of 280 observations of independent binomial random variables. Find the mle of \(p\).

\item {} 
Pooling the last three cells, test the agreement of the observed frequency distribution with the binomial distribution using Pearson’s chi-square test.

\item {} 
Apply the test procedure derived in the previous problem.

\end{enumerate}

Solution:
\begin{enumerate}
\item {} 
Suppose the number of breaks for each of the \(280\) bars in 5 cases is \(y_i\), for \(i = 1, \dots, 280\).

Under the null hypothesis, the mle of p is

\(hat{p}=\frac{157 \cdot 0 + 69 \cdot 1 + 35 \cdot 2 + 17 \cdot3 + 1 \cdot 4 + 1 \cdot 5}{5 \cdot 280} = 199/1400\)

\item {} 
The Pearson's chi-square statistics is

\(X^2 = \sum_{i=1}^m \frac{(x_i - n\hat{p})^2}{n\hat{p}}\)

which has degrees of freedom as \(6 - 1 = 5\).

Under the null hypothesis, \(p_k(\hat{p}) = {5 \choose k} \hat{p}^k (1 - \hat{p})^{5 - k}\), for \(k = 0, \dots, 5\). Or we have
\(p_0(\hat{p}) = (1 - \hat{p})^5 = 0.46460\)
\(p_1(\hat{p}) = {5 \choose 1} \hat{p} \cdot (1 - \hat{p})^4 = 0.38491\)
\(p_2(\hat{p}) = {5 \choose 2}\hat{p}^2 (1 - \hat{p})^3 = 0.12755\)
\(p_3(\hat{p}) = {5 \choose 3} \hat{p}^3 (1 - \hat{p})^2 =  0.02114\)
\(p_4(\hat{p}) = {5 \choose 4} \hat{p}^4 (1 - \hat{p})0.00175\)
\(p_5(\hat{p}) = \hat{p}^5 = 0.00005\).

Pooling the last three cells gives

\(X^2 = \frac{(17 - 280 \cdot 0.02114)^2}{280 \cdot 0.02114} + \frac{(1 - 280 \cdot 0.00175)^2}{280 \cdot 0.00175} + \frac{(1 - 280 \cdot 0.00005)^2}{280 \cdot 0.00005} = 90.71675\)

where the corresonding p-value is 0 which implies that we should reject the null hypothesis.

\item {} 
By the previous procedure, we have

\(X^2 = \sum_{i=1}^280 \frac{(x_i - 5 \cdot \hat{p})^2}{5 \cdot \hat{p}(1 - \hat{p})} = \frac{1}{5 \cdot \hat{p}(1 - \hat{p})} \cdot (157 \cdot (0 - 5 \cdot \hat{p})^2 + 69 \cdot (1 - 5 \cdot \hat{p})^2 + 35 \cdot (2 - 5 \cdot \hat{p})^2 + 17 \cdot (3 - 5 \cdot \hat{p})^2 + 1 \cdot (4 - 5 \cdot \hat{p})^2 + 1 \cdot (5 - 5 \cdot \hat{p})^2) = 429.0169\)

which is a chi-square statistics with degrees of freedom as \(280 - 1 = 279\). The corresponding p-value is 0. So it also implies that we should reject the null hypothesis.

\end{enumerate}


\chapter{Problem 5 (Section 9.11 Problem 43)}
\label{P5:problem5}\label{P5::doc}\label{P5:problem-5-section-9-11-problem-43}\begin{enumerate}
\item {} 
In 1965, a newspaper carried a story about a high school student who reported
getting 9207 heads and 8743 tails in 17,950 coin tosses. Is this a significant
discrepancy from the null hypothesis \(H_0: p = \frac{1}{2}\)?

\item {} 
Jack Youden, a statistician at the National Bureau of Standards, contacted the student and asked him exactly how he had performed the experiment (Youden 1974). To save time, the student had tossed groups of five coins at a time, and a younger brother had recorded the results, shown in the following table:

\noindent\begin{tabulary}{\linewidth}{|L|L|}
\hline
\textsf{\relax 
Number of Heads
\unskip}\relax &\textsf{\relax 
Frequency
\unskip}\relax \\
\hline
0
&
100
\\
\hline
1
&
524
\\
\hline
2
&
1080
\\
\hline
3
&
1126
\\
\hline
4
&
655
\\
\hline
5
&
105
\\
\hline\end{tabulary}


Are the data consistent with the hypothesis that all the coins were fair (\(p = \frac{1}{2}\))?
\begin{enumerate}
\setcounter{enumi}{2}
\item {} 
Are the data consistent with the hypothesis that all five coins had the same probability of heads but that this probability was not necessarily \(\frac{1}{2}\) ? (Hint: Use the binomial distribution.)

\end{enumerate}

\end{enumerate}


\chapter{Problem 6 (Section 9.11 Problem 44)}
\label{P6:problem6}\label{P6:problem-6-section-9-11-problem-44}\label{P6::doc}
If gene frequencies are in equilibrium, the genotypes AA, Aa, and aa occur with probabilities \((1 - \theta)^2\) , \(2 \theta (1 - \theta)\), and \(\theta^2\), respectively. Plato et al. (1964) published the following data on haptoglobin type in a sample of 190 people:

\noindent\begin{tabulary}{\linewidth}{|L|L|L|}
\hline
\multicolumn{3}{|l|}{\relax \textsf{\relax 
Haptoglobin Type
\unskip}\relax \unskip}\relax \\
\hline
Hp1-1
&
Hp1-2
&
Hp2-2
\\
\hline
10
&
68
&
112
\\
\hline\end{tabulary}


Derive and carry out a likelihood ratio test of the hypothesis \(H_0:\theta = \frac{1}{2}\) versus \(H_1: θ \neq 1\).



\renewcommand{\indexname}{Index}
\printindex
\end{document}
